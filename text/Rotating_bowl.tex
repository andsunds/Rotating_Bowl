\documentclass[11pt, a4paper, twocolumn, swedish, english]{article}
\pdfoutput=1


\usepackage{custom_as}

\graphicspath{ {Figures/} }
\usepackage[margin=10pt, font=small]{caption}
%%För att lägga in 'att göra'-noteringar i texten
\usepackage{todonotes} %\todo{...}, \todolist


%%För att själv bestämma marginalerna. 
\usepackage[
            top    = 3.5cm,
            bottom = 3.5cm,
            left   = 3cm, right  = 3cm
]{geometry}

%%För att ändra hur rubrikerna ska formateras
%\renewcommand\thesection{...}




\begin{document}

\title{Measuring $g$ using a rotating liquid mirror --
  Encouraging students' experimental creativity} 

\newcommand{\andsunds}{andsunds@chalmers.se}
\author{Andréas Sundström\footnote{Engineering Physics, Chalmers
    University of Technology, Gothenburg, Sweden,
    \textcolor{blue}{\href{mailto:\andsunds}{\nolinkurl{\andsunds}}} }
\and Tom Adawi\footnote{Engineering Education Research, Chalmers
  University of Technology, Gothenburg, Sweden} 
}
\maketitle

\begin{abstract}
    We describe a low-cost yet experimentally
    challenging method to measure the acceleration of gravity, $g$,
    using a rotating liquid surface and a laser pointer. The key idea
    underpinning this novel method is that a rotating liquid surface
    will form a parabolic reflector which will focus the light to a
    known focal point. By measuring the height of the focal point, $g$
    could be determined to $\unit[9.78 \pm 0.13]{m/s^2}$. We discuss the
    pedagogical merits of this method compared to more traditional
    methods for measuring $g$, and how it can be implemented as an
    experimental problem at different educational levels. 
\end{abstract}


\section{Introduction}

The acceleration of gravity, $g$, has been measured countless times in
secondary school physics laboratories. This is usually done through
some sort of pendulum or free fall experiment. In terms of
experimental \emph{simplicity} these methods are hard to beat. On the other hand, these methods do not excite much experimental \emph{creativity}. In these cookbook-style labs, the students follow step-by-step
instructions and there is little room for intellectual engagement. As
pointed out by Germann et al.~\cite{Germann1996}, ``Recipe-like activities often short
circuit opportunities to stimulate thinking by
students''. Consequently, cookbook labs have very limited impact on
students' understanding of the nature and practice of
science~\cite{Domin1999}. 

In this paper, we describe a novel and more thought-provoking way to
measure the acceleration of gravity. The main idea of the method is to
study the surface of a liquid in a rotating bowl. The surface will
form a \emph{parabolic reflector}~\cite{Berg1990}, which will focus
incoming planar light. In the next section we show that the focal
point of the parabolic reflector is located at the height
\begin{equation}
h=\frac{g}{2\omega^2}
\end{equation}
above the vertex, where $\omega$ is the angular
frequency of the bowl. By locating the focal point with a laser pointer, the value of $g$ can be determined.
Since only the focal height, $h$, and the angular
frequency, $\omega$, will influence the value of $g$ this method
is experimentally quite clean, with only two parameters to measure. 

Since there are many ways to locate the focal point, for example, 
the proposed method can stimulate the students' experimental creativity.
Moreover, the students are encouraged to integrate their knowledge 
from two areas of physics -- mechanics and optics. Michael Faraday 
said: ``I was never able to make a fact my own without seeing it''.
Hence an additional pedagogical benefit of the proposed method, is that students
actually get to \emph{see} the focal point of a parabolic reflector (see Figure 2).
In this way, the proposed experiment to measure $g$ addresses the three fundamental objectives
of laboratory instruction as defined by Ernst:
``First, the student should learn how to be an experimenter.
Second, the laboratory can be a place for the student
to learn new and developing subject matter. Third, laboratory
courses help the student to gain insight and understanding of
the real world''~\cite{Ernst1983}.

A somewhat similar method for measuring the acceleration of gravity has been used as part of an experimental
problem in the International Physics Olympiad,
IPhO~\cite{IPhO2001}. However, in the IPhO problem a
different aspect of the parabolic surface was used
to determine $g$. They used the fact that the surface, in a rotating cylindrical vessel, will not change its height at a certain distance from the center, independent of rotational speed. Apart from this, we have not been able to find any previous reports of experiments that use a rotating liquid surface to
determine the acceleration of gravity.

\section{Theoretical background}
First we find the equation for the surface in the rotating bowl. Then we use the equation to prove that a parabola will have a focal point, and in the process obtain a relationship between the height of the focal point and $g$.

\begin{figure}\centering
\resizebox{1\linewidth}{!}{\input{Figures/vattenparabel.pdf_t}}
\caption{Cross section of the liquid surface in a rotating vessel. The
  surface will be perpendicular to the net acceleration in the
  rotating frame of reference. Since the centrifugal acceleration will
  increase linearly with $r$, so will the slope of the surface
  ${\nicefrac{\dd{z}}{\dd{r}}=\tan\alpha}$. }
\label{fig:parabola} 
\end{figure}



By studying the centrifugal acceleration in the rotating frame of reference alongside gravity, we see from \figref{fig:parabola} that the 
angle $\alpha$ of the surface depends on $g$ and the centrifugal
acceleration $a_c$. To obtain the full relationship, we first note
that the surface of a freely flowing liquid will be perpendicular to
the perceived net acceleration. Then we note that the derivative of a
function is equal to the tangent of the angle to the horizontal line, whereby we can get an
expression for the derivative of the slope of the surface in
\figref{fig:parabola} 
\begin{equation}\label{eq:slope}
\dv{z}{r} = \tan\alpha = \frac{a_c}{g}.
\end{equation}
With the centrifugal acceleration $a_c=r\omega^2$, the equation
for the surface can be found by integrating the left and right hand
side of~\eqref{eq:slope}: 
\begin{equation}\label{eq:parabola}
z(r)=\frac{r^2\omega^2}{2g}.
\end{equation}
In 2010 \v{S}abatka and
Dv\v{o}rák~\cite{Sabatka2010} also confirmed this experimentally using
a set of metal rods. 

The fact that the surface is parabolic can then be used optically. A
parabolic mirror will reflect any vertically incoming light to the
focal point of that parabola. This property was used
by~\cite{Berg1990} to focus light into an optical image. There's even
a telescope, the Large Zenith Telescope~\cite{LargeZenith}, which uses
a rotating mercury mirror. 

\begin{figure}\centering
\resizebox{.55\linewidth}{!}{\input{Figures/angles.pdf_t}}
\caption{The watersurface with the incoming and reflected light. Since
  the angles in the figure sums to
$\alpha+\beta= 90^\circ = \gamma+\beta$, the angle between the incoming
light and the normal to the surface will be~$\gamma=\alpha$. }
\label{fig:angles} 
\end{figure}

\begin{figure*}
\centering
\resizebox{.6\linewidth}{!}{\input{Figures/rot_bowl.pdf_t}}
\caption{Schematic drawing of the setup together with the definitions
  of the notation used in the calculations. Due to the focal height
  being measured slightly outside the exact center of the rod, by
  $\rho$, a small correction $\delta$ has to be made to the measured
  focal height.} 
\label{fig:rot_bowl} 
\end{figure*}

To use the focusing property of a parabola, the height of the
focal point has to be known. To find this, we begin by studying
\figref{fig:angles}. There we see that, since we have the two right
angles, between on one hand the incoming ligt and the horizontal line
and on the other hand between the surface and the normal, both
$\alpha+\beta = 90^\circ$ and $\gamma+\beta = 90^\circ$. Therfore the
total angle between the incoming and refelcted light is $2\alpha$. 

Furthermore, we see from \figref{fig:rot_bowl} that the point where the
reflected light crosses the axis of rotation is at the height 
\begin{equation}
h= z(r) + r\,\cot(2\alpha)
\end{equation}
above the vertex of the parabola.
To show that this will be a focal point, we have to show that $h$ is
independent of $r$. To do this, we can use \eqref{eq:slope} and
\eqref{eq:parabola} together with the trigonometric identity
\begin{equation}
\begin{aligned}
\cot(2\alpha)
&=\frac{\cos^2\alpha-\sin^2\alpha}{2\,\cos\alpha\,\sin\alpha}\\
&=\frac{\cos\alpha}{2\sin\alpha}-\frac{\sin\alpha}{2\cos\alpha}\\
&=\frac{1}{2}\left(\frac{1}{\tan\alpha} -\tan\alpha \right),
\end{aligned}
\end{equation}
to obtain
\begin{equation}
h=\frac{r^2\omega^2}{2g} 
   + \frac{r}{2}\left(\frac{g}{r\omega^2} -\frac{r\omega^2}{g} \right)
=\frac{g}{2\omega^2}.
\end{equation}
Here we find indeed that $h$ is independent of $r$ (and $\alpha$),
whereby the proof that a parabola will have a focus is complete. We
can also note that this expression can be rewritten to get
\begin{equation}
g=2h\omega^2,
\end{equation}
which will be used to measure $g$. 



\section{Experimental setup}

The experimental set up shown in \figref{fig:rot_bowl_pic} consists of a bowl of water
on a spinning disk, a laser pointer, a central rod marking out the axis of
rotation, and a photo-diode for measuring the speed of rotation. The spinning
disk was powered by a DC motor so the rotational speed could be
adjusted by varying the voltage applied to the motor. 

The focal point will be located somewhere along the axis of rotation. When
shining a laser vertically down at the parabolic surface, the refection will hit
the central rod at the height $h$ over the vertex, which can be measured
using a narrow ruler. 

By varying the rotational speed and measuring the focal height for
different speeds, a linear regression could be made for $g$ as the
slope of the curve of $h$ plotted against $1/(2\omega^2)$. 
Measurements were taken at two different radial distances from the
center to verify that the radial distance has no impact on the measured
value of $g$. 

The rotational speed, $\omega$, was measured with a photo-diode connected
to a digital oscilloscope which could record several periods of
rotation. A more readily available option is to connect the
photo-diode to a computer's microphone input and record the signal as a
``sound'' file on the computer, as in~\cite{Sabatka2010}. Since
$\omega$ is one of the two parameters directly determining the value
of $g$, it's recommended to measure the periods with some kind of
digital logging device, such as an oscilloscope or computer, to
minimize uncertainty in this parameter.

\begin{figure}
\centering
\resizebox{6cm}{!}{\input{Figures/rot_bowl_pic.pdf_t}}
\caption{A photograph of the setup running. The
  reflected beam from a laser pointer (outside frame) can be seen at
  the focal point on the central rod. } 
\label{fig:rot_bowl_pic} 
\end{figure}

\subsection{Calibration}

The laser pointer has to be perfectly vertical. This calibration can be
done by shining the laser down on the water surface when it's not
rotating. The laser-beam will be perfectly vertical when the
reflection hits the source.

To get the central rod centered, a center mark was put in the rotating
disc. A spirit level was used to ensure that the central rod was
vertical. Both of these steps are essential, since deviation from the
center will change the measured value of $g$, as we will see in next
section. 


\subsection{Correction}\label{sec:corrections}

As shown in \figref{fig:rot_bowl}, the measured focal height
$\hat{h}$ is slightly off from the real focal height by 
\begin{equation}%\label{eq:correction}
\delta=\rho\cot 2\alpha
\end{equation}
due to a finite width, $\rho$, of the central rod. This width is easily
measured with a pair of calipers.

We now have to determine $\alpha$. First, we remember that
\begin{equation}\label{eq:dz/dr}
\tan\alpha=\dv{z}{r} =\frac{2 z(r)}{r}.
\end{equation}
Then, we can see from \figref{fig:rot_bowl} that 
\begin{equation}
\cot 2\alpha =\frac{\hat{h} - z(r)}{\hat{r}} 
= \frac{\hat{h}-\frac{r}{2}\tan\alpha }{\hat{r}},
\end{equation}
where we used \eqref{eq:dz/dr} to substitute $z(r)$ in terms of
$\tan\alpha$. By rewriting the last equation, we get
\begin{equation}
\cot 2\alpha 
-\frac{\hat{h}}{\hat{r}}
+\frac{\hat{r}+\rho}{2\hat{r}}\tan\alpha  = 0.
\end{equation}
This expression consists of known or measured quantities and $\alpha$. From
this stage, $\alpha$ can easily be obtained to a satisfying degree with
regular numerical equation solvers.


\section{Results and discussion}

The results from the measurements together with the least square fit
are shown in \figref{fig:data}. The calculated value of the
acceleration of gravity comes out as
\begin{equation}
g=\unit[9.78\pm 0.13]{m/s^2}.
\end{equation}
The uncertainty given is the standard error in the mean.

\begin{figure*}\centering 
\includegraphics[width=.8\linewidth]{g_minsta_kvadrat.pdf}
\caption{\label{fig:data} Measurements of the focal height $h$ (with
  corrections for a center rod of non-zero width) plotted against
  $1/(2\omega^2)$. The acceleration of gravity $g$ is given as the
  slope of the least square fit to these data points. Measurements
  were taken at two different radial distances, $r$, from the center.
}
\end{figure*}

Since this is a new method for measuring $g$, it is not
easy to compare our result to those by others. The closest
comparison we can make is to the results from the experimental part in 
the IPhO in 2001~\cite{IPhO2001}. There, it is stated that $g$ could be measured to $\unit[9.8]{m/s^2}$ with a standard error of $\unit[0.2]{m/s^2}$. So our result seems to be at least as good, or perhaps even a little better, compared to those obtained through other methods using a rotating liquid. 

The error in our method is, however, considerably larger than
the error in some of the most sophisticated pendulum methods~\cite{Candela2001}. 
But as mentioned in the introduction, the proposed method has several pedagogical advantages
compared to more traditional methods for measuring $g$. Apart from encouraging
experimental creativity, this method encourages students to consider experimental corrections. 

The significance of the correction, $\delta$, due to the finite width of the central
rod is quite notable. The value of $g$ comes out to be only
about $\unit[9.2]{m/s^2}$ \emph{without} these corrections. This
demonstrates the necessity of the calculations made in
section~\ref{sec:corrections}. 

\subsection{Experimental improvements}

Most of the uncertainty in the result is due to small ripples on the
water surface, leading to a jittery motion of the reflected
laser beam. Both Berg~\cite{Berg1990} and the IPhO~\cite{IPhO2001} experimental problem used
glycerin as the liquid. Due to the high viscosity
of glycerin, there will be less ripples on the surface.

Another way to increase the accuracy would be to use a bigger bowl,
preferably one with vertical walls. The maximal rotational
speed will be limited by the slope of the walls, since the walls have
to have a steeper slope than the water surface.

Finally, there are other ways to locate the focal point. For example, two
different lasers could be used. By shining the two laser beams
vertically down at two different points at the rotating surface, the
focus can then be located at the point of intersection of the reflected
beams. Note that this will eliminate the need to calculate the correction in
section~\ref{sec:corrections}. 

\subsection{Pedagogical reflections}

The experiment to measure the acceleration of gravity described in
this paper could be carried out as a shorter project for secondary
school students or as a laboratory exercise for first-year
undergraduate physics students. One should note though, that the
experiment takes around 2--4~hours from scratch to finished
measurements, depending on the desired experimental accuracy. If this
experiment is conducted as a project, the students could compare this
method to more traditional methods for measuring the acceleration of
gravity.

Traditional laboratory instruction has been criticized for the
emphasis it places on following step-by-step instructions, leaving
little room for intellectual engagement~\cite{Domin1999}. In these
cookbook labs, ``virtually no attention is given to the planning of
the investigation or to interpreting the
results''~\cite{Domin1999}. As a consequence, ``virtually no
meaningful learning takes place''~\cite{Domin1999}. As a way to avoid
the cookbook-style lab exercise, the method for measuring the
acceleration of gravity described in this paper could form the basis
for a more open experimental problem. The students could, for example,
be asked to determine the acceleration of gravity using some given
equipment. But as always when there is no ``recipe'' to follow,
unintended solutions might appear – like the one
in~\cite{IPhO2001}. This is, however, not necessarily a bad thing. As
Pickering noted, in cookbook labs, students are never ``forced to
reconcile results, or confronted with challenge to what is naively
predictable''~\cite{Pickering1987}.

The instruction to determine something using a given set of equipment
and with no further information regarding the procedure, is common in
physics competitions, like the IPhO. An experimental problem with this
type of minimal instruction, might therefore be more suited for
students who already have a good understanding of physics. Although
this type of minimal instruction is common in the IPhO, the
experimental problem in 2001~\cite{IPhO2001} included quite a few
hints. This indicates that the concept of measuring the acceleration
of gravity with a rotating liquid surface could be too challenging for
most secondary school students without additional guidance. The hints
provided in the IPhO experimental problem in 2001 could be used as a
source of inspiration for developing laboratory instructions suitable
for students with different levels of background knowledge and at
different educational levels. 

\section{Conclusion}

In this paper, we have described a novel method for measuring the acceleration of gravity. 
As an \emph{experimental} method, it fares well compared to similar methods~\cite{IPhO2001}, 
but not as well compared to some of the most accurate methods~\cite{Candela2001}. 

As a \emph{pedagogical} method, it holds great potential. 
The overall aim was to break away from the traditional cookbook-style laboratory exercise and engage students in more 
authentic scientific thinking. With suitable scaffolding,
the proposed method allows students to integrate knowledge from two areas of physics, mechanics and optics,
and develop important skills, including experimental creativity. 
Moreover, students get to see the focusing property of parabolic mirrors while doing this experiment.

\section*{Acknowledgments}

We are very grateful to the personnel in the machine shop at the
Physics Department at Chalmers. We thank Lars Hellberg and Jan-Åke
Wiman for constructing and balancing the rotating disk so that the
experiments could be performed. 

%Bibliography
\bibliographystyle{ieeetr}
\bibliography{bibliography}

\end{document}